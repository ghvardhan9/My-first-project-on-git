\documentclass{beamer}

\usepackage[utf8]{inputenc}

\title{Matrix Approach To Find Orthocenter}
\author{EE18BTECH11018   EE18BTECH11020}
\institute{IIT HYDERABAD}
\date{14.02.2019}

\begin{document}

\frame{\titlepage}

\begin{frame}
\frametitle{Geometric Question}

\label{prob:draw_triangle}

Q). Find Orthocenter of triangle with given vertices

      
      
      A = (2,-6)
   
      B = (5,2)
   
      C = (-2,2)


            

Let us take Transformation matrix

\begin{equation}
\vec{A} =
\begin{pmatrix}
-2
\\
-2
\end{pmatrix},
\vec{B} =
\begin{pmatrix}
1
\\
3
\end{pmatrix},
\vec{C} =
\begin{pmatrix}
4
\\
-1
\end{pmatrix}
\end{equation}
Now let the altitudes AD and BE  meet at H. we need to find H.

AD :  [B - C]^T(x - A) =  0

BE :  [A - C]^T(x - B) =  0
\end{frame}

\begin{frame}
\frametitle{Matrix Transformation}


Let the equations of AD and BE  are respectively:


\begin{align}
\vec{n}_1^T\vec{x} &= p_1 \text{  and}
\\
\vec{n}_2^T\vec{x} &= p_2
\end{align}


where  p_1  =  [B-C]^T\vec{A}


 p_2  =  [A-C]^T\vec{B}
      
      
since AD \perp BC


Now 

   Normal vector of AD  =   directional vector of BC  =  B - C


similarly


  Normal vector of BE  =   directional vector of AC  =  A - C
  
  

where {n_1}  \\  -  the normal vector of AD  =  (B - C)
 
and {n_2}    \\  - the normal vector of BC   =  (A - C)


    
  

\end{frame}   
   
\begin{frame}{Solution in Matrix form}


Other wise we have :


    {n_1}  =  B - C
    
    
    {n_2}  =  A - C
    
    

Combining the both equations in the form of matrix we get


\begin{align}
\myvec{ \vec{n}_1^T
\\
\vec{n}_2^T}
\vec{x}
&=
\vec{p}
\\
\implies N^T \vec{x}
&=
\vec{p}
\end{align}

\end{frame}

\begin{frame}
\frametitle{Solution to Matrix form}

where
\begin{equation}
N = \myvec{ \vec{n}_1 &
\vec{n}_2} =  [B-C , A-C] = 
\end{equation}
%
The point of intersection is then obtained as
\begin{align}
\vec{x} &= \brak{N^T}^{-1}  \vec{p}
\\
&= N^{-T} \vec{p}
\end{align}
%
The following  yields the point of intersection 
\begin{equation}
\vec{H} =
\begin{pmatrix}
2
\\
0.5
\end{pmatrix}
\end{equation}


following the same process for the intersection of BE and CF gives the same point



\end{frame}

\end{document}

